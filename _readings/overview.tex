
\textit{Beating the Worlds Best at Super Smash Bros. Melee with Deep Reinforcement Learning}
@article{firoiu2017beating,
  title={Beating the World's Best at Super Smash Bros. with Deep Reinforcement Learning},
  author={Firoiu, Vlad and Whitney, William F and Tenenbaum, Joshua B},
  journal={arXiv preprint arXiv:1702.06230},
  year={2017}
}

Very recent paper, not cited.

Looks directly at the memory of the game. 
This is actually accessing the memory of the game instead of behaving more like a human (e.g. looking at graphics and controller inputs).
In my work we will try harder to look at the human equivilent perspective.

They do NOT use a Deep Q network (DQN).
This is a TD method of machine learning.

In this paper, there are two reinforcement learning approeaches tried to compete at Super Smash Bros. Melee (SSBM).
This game is very similar to the game we are working with.
Their approach to data collection is very similar to ours, except they explicitely do not use the buffer information.
They instead get precise information from memory space.

They use two RL techniques, Q-Learning (which seems to be learning the 'value-function'.
Second they use the 'policy-gradient' method which is referred to as 'actor-critics'.



\textit{https://en.wikipedia.org/wiki/Reinforcement_learning}

There is some \textit{reward} for completing an action well.
We can take \textit{actions} in some \textit{environment}.

Reinforcment Learning (RL) is also known as \textit{approximate dynamic programming}.
RL is typically modeled as a Markov Decision Process (MDP) -- this indicates that some inputs into the system are _not_ under the control of the machine learning tool or may be random.
In our case this is seen as the inputs from the oponent.

This is different from Supervised Learning since ``Correct inputs and outputs are never presented, nor are suboptimal solutions explicitely flagged/corrected".
Additionally, there is an implicit focus toward ``online learning" which is being able to do the learning as the process is running.

Basic RL algorithm and overview:
1. A set of environment and agent states "S"
2. set of possible actions "A"
3. Policies for transitioning from state to actions (this will be something like a full-mesh finite state machine for SSB).
4. Rules that determine the \textit{scalar immediate reward} of a transition 
  (this is the prediction rule).
5. Rules that describe the actual observation 
  (this is the actual observation).

I think that (4) is the prediction, e.g. we want to maximize the scalar immediate reward!

I think that we can essentially ignore (3) since we have the abillity to take all actions at any time, right?
NOTICE: What about when we are airborn, are we allowed to jump at that time?
Certainly we are allowed to press the button but it will not have any effect... It will have the same scalar immediate reward as not moving.

Notice that we can perform \textbf{multiple actions per time slice} in a video game.
I.e. holding 'right' while pressing attack.
I.e. holding 'left' while pressing jump.
Instead of saying that we perform multiple actions, we should say that we can perform _compound_ actions. 
This makes it so that our possible actions are the power-set of the available independent actions.

Basic algorithm:
  \begin{listlisting}
Receive Observation at time t:
  o_t, r_t (input e.g. frame buffer, and reward, e.g. damage delt)
  a_t : Choose an action from set of available actions.
  the simulation advances given this action
  \end{listlisting}

The idea is that the algorithm will choose an action using all of history as its learned base.
This makes it so that there could be a action that is worse for the next moment but has a long-term pay-off.
Analogy: Deciding to make a locally bad decision toward a long-term good decision is like going to school -- it will pay off financailly in the long run even at a cost now.

RL uses samples to optimize performance.
RL uses function approximation to deal with large environments.

\textit{RL key problem: find out which actions are good based on past experience.}

Discounting is kind of like 'history weights' I think.

A lot of RL assumes a fixed policy, e.g. we are just testing a policy in the provided world.

Temporal Difference methods allow policy to change over time.
  Issues is that there might not be convergence onto a single policy.
 

  \textit{Summary of "Playing atari with Deep Reinforcement Learning"}
There are multiple applications of reinforcement learning to generate models and train them to play video games, such as Atari games. 
In these models, reinforcement learning finds the actions with the best reward at each play. 
This method is a widely used method in combination with deep neural networks to teach computers to play Atari video games.

This paper demonstrates that a convolutional neural network can overcome these challenges to learn successful control policies from raw video data in complex RL environments.

Seems like the convolution neural network aspect does not abstractly match our project very well -- it matches the atari project very well but not the n64 state-reading problem.
The difference is that their project focuses on using raw inputs, our project has a feature extraction/preprocessing aspect associated to it.
Convolution NN are essentially for image recognition problems.

It is not at all yet clear how "Q-learning" and "stochastic gradient descent" are composed.
The latter is back-propogation in a ANN, the former is a DP technique for constructing an optimal policy.
[10]


\textit{https://en.wikipedia.org/wiki/Convolutional_neural_network}
A simple feed forward neural network.
Connectivity pattern is inspired by animal's visual cortex (cool).
This style of neural network seems specially built for projects where we are working on a visual input (i.e. a screen buffer).

\textit{https://en.wikipedia.org/wiki/Temporal_difference_learning}

A mix of Dynamic Programming and Monte Carlo methods.

Premise: TD learning considers that subsequent predictions are often correlated in some sense.

When used for RL, (our case), TD relies on the Bellman Equation (From bellman-ford Dynamic Programming, right?)

Seems to be just a mathematical model for predicting reward of a given action. :D


\textit{Sutton Book}

CHAPTER 2:

\quote{It is a striking feature of the reinforcement learning solution that it can achieve the effects of planning and lookahead without using a model of the opponent and without conducting an explicit search ofver possible sequences of future states and actions.}


CHAPTER 3:

``Anything that the actor cannot directly and arbitrarily modify is considered a part of the environment."
This makes it obvious that the AI players location is part of the enviornment since it does not have \textit{arbitrary} control of the location.
The agent-environemnt boundry represents the limit of the agent's absolute control, not of its knowledge.

Notice that the 'environment' in the Reinforcement Learning model is anything that the player does not 100\% directly control.
It might be more fruitful to look only at the opponent's bitmap and to not worry about our character's bitmap.

This model is: ``Goal directed learning from interaction"

``State can be a data structure, same with actions. However, rewards are always single numbers."
Looks like there can be multiple inputs into the single reward though.
E.g. perform a minor negative reward when something goes a little bit wrong.

EX 3.1:
tasks that can be modeled using the reinforcement learning framework:

1. brushing teeth
actions -- motor controls
environment -- whiteness of teeth, pain receptors, dentist bills
reward -- Joy: negative when receiving bills, slightly positive when getting complimented on smile

2. Making a sandwich
actions -- get ingredients, put down an ingredient
environment -- how much food consumed, how tasty is the food, how much food wasted, how messy is the food
reward -- Joy

3. Finding a Job
actions -- apply, interview, negotiate
environment -- market, 

4. Fighting Robot
actions -- punch, block, move
environment -- opponent, room, damage dealt, kos
reward -- super positive for KOs, slightly positive or negative for damage ratio

EX 3.2:
When the reward cannot be quantified, we cannot use this framework.

EX 3.3:
Just depends on what we want the reward to be based on -- do we want to get to a location (i.e. guidence) then we should be at the highest level of abstraction.
Do we want to have the reward be related to not getting in a collision? Then we should probably be at the lower level.
Do we want to have the reward be the minimal effort/stress for the driver? then we should be at the muscular and Computer vision level.


Section 3.2:

Reward is a simple number for each timestep.
Goal is to maximize the Total amount of reward recieved.

What if we make it so that getting a KO gives 350 reward.
Each point of damage is worth 1 reward.
Each health lost is worth -1 reward.
Each life lost is -350 reward.

NOTICE: The reward signal is \textbf{not} the place to impart knowledge about \textit{how to achieve the end goal}.
I.e. by this we should \textbf{absolutely not} use SSB damage as a reward.

Section 3.3: 

How do we actually maximize on the reward?
SSB can be thought of as an episodic task where each episode is a KO?
Or where each episode is an entire gameplay session.

Total Reward (return) can be calculated very simply as the sum of all reward.

For some reason, we use a discounted reward.
Where each n+1 timestep is worth a little bit less than the nth timestep...
Using an exponential decay parameter known as the \textit{discount rate}.

The closer the discount rate is to '0', the less it cares about the future.
The closer the discount rate is to '1', the more farsighted it is.

\textbf{question: Would there be any reason to use discounting if we are epsodic?}


Section 3.5:

\textit{Markov Property}: A state signal that succeeds in retaining all relevent information is said to be Markov.
In a game this means that the state describes all relevent information to win the game.
I.e. all of the locations on the board for a game of checkers.
``Independence of path" property -- all that matters is in the current state signal.

We assume the Markov Property -- but we know that it is not completely true unless we encode all of the history in our state.
What if we encode just some of the history into our state, like when a particular move started?

\textbf{NOTICE:}
To get closer to the markov property, it would be super duper helpful to have the state maintain "all recently received inputs."

Section 3.6

From this book we have gathered a reference for basis knowledge for all reinforcement learning methods.
This book will be an invaluable reference for anyone implementaing RL.
Notice however that this book covers techniques but not implementations -- at its core this is a book of the academia behind RL and less helpful toward building a RL ANN algorithm.


\textit{https://arxiv.org/pdf/1507.04296.pdf}
cited by 45

Seems to be a harness where there 
