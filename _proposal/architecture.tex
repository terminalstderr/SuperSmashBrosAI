\chapter*{Highlevel Architecture}

All development was be performed on the Windows platform using Visual Studio 2015.
We have based our project off of the open source Project64 emulator.

Below is the overlying software architecture which indicates the data flow.

\ssbFigure{SSBAI_dll.png}

We explode this data flow to reveal critical tasks that are completed per frame.

\ssbFigure{SSBAI_tasks.png}

\section*{The Hook: Emulator Modification}
First, we inserted one simple hook was into the emulator's per-frame-update function. 
We used this hook to get a pointer to the N64 memory space. 
This provided freedom to access any in-game data structures in real time, at each time step. 

\section*{Finding N64 Data Structures}
Determining the memory offset of interesting data structures within the emulators address space was performed using Cheat Engine.
% http://www.cheatengine.org/
Usage of this tool can be summarized as 
\begin{enumerate}
\item reviewing some range of the process' memory,
\item  running the process for a step performing some distinct action, then
\item narrowing the investigated memory range using a byte-wise diff.
\end{enumerate}

More precisely, this tool acts as a debugger with a rich interface and a set of features that easily enable finding particular emulated data structures.
The tool is performing a sequence of diffs over the virtual memory space of the debugged process as the process is running.
Each diff reduces the amount of virtual memory space being considered until eventually there are only a handful of addresses that are correlated to the distinct action being performed.

As an example, to find the memory offset associated to the location of a character:
\begin{enumerate}
\item Run an initial scan over the N64 virtual address space
\item Move the character to the left in the emulator
\item Run a diff in Cheat Engine
\item Repeat step 2 and 3 until the relevant/correlated virtual address space is only a few addresses.
\end{enumerate}
At this point, we have determined the memory address that is correlated to the action of the character moving.
Doing some investigation of the correlated memory addresses and nearby memory spaces, we easily find the address corresponding to the floating point values of the characters x and y position!

\section*{Extracting State Values}

State extraction was performed using very crude memory manipulation.
A set of macros was defined for reading arbitrary memory into our data structures.
Issues arose with correctly determining the memory offset: original calculations of the memory offset were incorrect which was only discovered between successive runs after restarting Visual Studio.
The ultimate solution was to find the relevant data structure and get the offset of that from the emulated N64 memory address all from within Visual Studio.
The result is that state gathering works across all development machines successfully.

Normalization of the state was also performed.
This was simply moving all data members to the range of $(-1.0, 1.0)$.
We used domain specific information (e.g. SSB map limits, SSB max damage) to get the normalization range.
We also clipped the values in case somehow the limits are exceeded during game play to ensure clean processed data.

\section*{Enacting Actions}

Pushing an action into the emulator was trivial using a unioned bit-field.
The controller data structure in the emulated N64 memory space is provided by the N64 emulator, so there was no need to hunt down the data structure via cheat engine.

The possible action space was originally $2^{30}$, we limited our scope to $2^7$. 
Only combinations of the following actions are permitted:
\begin{itemize}
  \item Attack
  \item Special
  \item Shield
  \item Jump
  \item Sprint/Walk
  \item Left/Right
  \item Up/Down
\end{itemize}
