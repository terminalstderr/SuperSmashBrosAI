
\documentclass[12pt]{article}
\usepackage[margin=0.9in]{geometry}
\usepackage{graphicx}
\usepackage{csquotes}

\graphicspath{ {figs/} }

\title{Project Proposal: Parallel AI for ``Super Smash Bros. N64"}
\author{
        Ryan Leonard \\
        Boyana Norris \\
        CIS 531: \\
        Parallel Computing
}
\date{\today}

\begin{document}

\maketitle

\section{Overview}
We will pit a player against our AI at .

\subsection{Assumptions}
Kirby is our AI Player and champion.
The human is the evil Donky Kong.
We are playing on Kirby's level.
The only moves that Player and AI can perform is to move left, move right, and basic punch. 
%Kirby is honorable so he will also obey these rules.

\subsection{Computer Vision}
\begin{enumerate}
\item Detecting each players position
\end{enumerate}

\subsection{Machine Learning}
\begin{enumerate}
\item Determine the enemies probable move in the next $k$ frames
\item Rate each of my possible moves for the next frame.
\item Notice that the prior point it might be intelligent to rate the next $k$ frames and use that information with a weighted mean function that makes it so that the next frame best move is determined using the possible ranking of the move made 10 frames from now.
\end{enumerate}

\subsubsection{Input Data}
A sequence/window of 100 frames up until some physical exchange.
Each frame will consist of:
\begin{itemize}
\item AI Position
\item AI Velocity
\item AI Movement Input
\item AI Attack Input
\item Enemy Position
\item Enemy Velocity
\item Enemy Movement Input
\item Enemy Attack Input
\end{itemize}

\subsubsection{Training Labels}
Damage ratio is going to be our label for each datum.
We calculate this in the same sliding window fashion as well, i.e. last 100 frames. 
We can make the label modal by saying there is 4 different labels:
\begin{tabular}{|c|c|l|}
Label & Damage Ratio & Note \\\hline
kick-ass & $>10$ & AI is doing excellent \\
kicking & $>0$ & AI is doing well \\
kicked & $<0$ & AI is doing poorly  \\
ass-kicked & $<10$ & AI is doing horribly 
\end{tabular}

\section{Discussion}
\subsection{Getting access to controller, graphics, etc.}
We will look at the Project64 plugins and tweaking them to get access to these components.
If we can find an open source version of the plugins that would make it much easier to pass all of our outputs up.
It is perfectly acceptable to look at the inputs directly from the controller.

\begin{verbatim}

          +-------+
          |AI Eng.| 
          +-------+
          ^      ^
         /        \
  +--------+       \
  |Graphics|        \
  +--------+         v
      ^         +---------+   
       \        | Control | 
        \       +---------+
         \          ^
          \        /
           \      /
            v    v
          +-------+
          |  N64  | 
          +-------+
\end{verbatim}
Our AI Engine takes inputs from Graphics (i.e. it gets information from each frame for Computer Vision).
Additionally, inputs may be take from the control plugin to get more concrete information about 
Notice that the only inputs that our AI Engine has is into 


\section{Considerations}
Will consider buying an N64 controller for PC.

\end{document}
